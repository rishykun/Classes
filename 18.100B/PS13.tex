\documentclass{article}
\usepackage[margin=1.0in]{geometry}
\usepackage{amsmath}
\usepackage{amsfonts}
\usepackage{enumerate}
\setlength\parindent{0pt}
\begin{document}
\setlength{\abovedisplayskip}{0pt}
\setlength{\belowdisplayskip}{0pt}
\setlength{\abovedisplayshortskip}{0pt}
\setlength{\belowdisplayshortskip}{0pt}
\title{18.100B - Problem Set 13}
\author{Rishad Rahman}
\date{}
\maketitle
\begin{enumerate}
\setcounter{enumi}{9}
\item The hint basically shows everything necessary so I will just verify each step. The first step follows from the Fundamental Theorem of Arithmetic and the fact there will exist a term, $p=p_1^{e_1}p_2^{e_2}p_3^{e_3}...$ in the product expansion such that $\frac{1}{N} < \frac{1}{p}$. The second step follows from the formula for the sum of a geometric series. The third step follows from showing $(1-x)^{-1} \leq e^{2x}$ for $0\leq x \leq \frac{1}{2}$. Notice that $e^{2x} \geq 1+2x$ so $e^{2x} - \frac{1}{1-x} \geq 1+2x - \frac{1}{1-x} = \frac{x(1-2x)}{1-x} \geq 0$ when $0\leq x \leq \frac{1}{2}$. Then the result follows from comparison. \\
\setcounter{enumi}{11}
\item
\begin{enumerate}[(a)]
\item $c_n = \displaystyle \frac{1}{2\pi} \int_{-\pi}^{\pi} f(x) e^{-inx} dx = \frac{1}{2\pi} \int_{-\delta}^{\delta} e^{-inx} dx = -\frac{e^{-inx}}{2\pi in} \bigg|_{-\delta}^{\delta}=\frac{e^{\delta n i} - e^{ -\delta n i}}{2\pi n i}=\frac{\sin{n\delta}}{\pi n}$ for $n\neq 0$. $c_0=\frac{\delta}{\pi}$ obviously.\\
\item Note that $c_n = c_{-n}$ so $\displaystyle\sum_{n\geq 1} c_n =\sum_{n=1}^{\infty} \frac{\sin{n\delta}}{\pi n}= \frac{f(0)-c_0}{2} = \frac{1-\frac{\delta}{\pi}}{2}$ and multiplying across by $\pi$ gives the claim.\\
\item Note that $|f(x)|^2=f(x)$. By Parseval's theorem, $\displaystyle\frac{\delta^2}{\pi^2} +\sum_{n \neq 0}^{} \frac{\sin^2 n\delta}{\pi^2 n^2} = \frac{1}{2\pi} \int_{-\pi}^{\pi} |f(x)|^2 dx = \frac{1}{2\pi} \int_{-\pi}^{\pi} f(x) dx = c_0 = \frac{\delta}{\pi}$. So similarly we have $\displaystyle\sum_{n=1}^{\infty} \frac{\sin^2 n\delta}{\pi^2 n^2} = \frac{\frac{\delta}{\pi}-\frac{\delta^2}{\pi^2}}{2}$. Multiplying by $\displaystyle\frac{\pi^2}{\delta}$ gives the claim.\\
\item Note that the integral converges since it exists from $x=0$ to $x=1$, as there is a discontinuity of the first kind at $x=0$, and from $x=1$ to $\infty$ as we can compare to $\int \frac{1}{x^2}$ which does exist. Notice that the sum in (c) is uniformly convergent by comparison and so continuous so we can take $\delta\rightarrow 0$ using $\delta_n = \frac{1}{n}$ and it converges to $\frac{\pi}{2}$ (Theorems 7.11 + 7.12). So after substitution and taking limits,$\displaystyle \frac{\pi}{2} = \lim_{n\rightarrow\infty} \sum_{k=1}^{\infty} \frac{\sin^2 (\frac{k}{n})}{\frac{k^2}{n^2}} \frac{1}{n} = \int_{0}^{\infty} \left(\frac{\sin x}{x}\right)^2 dx$ since it is a Riemann sum. \\
\item $\displaystyle\sum_{n=1}^{\infty} \frac{\sin^2 n\frac{\pi}{2}}{n^2\frac{\pi}{2}} = \frac{2}{\pi} \sum_{n=1}^{\infty} \frac{1}{(2n-1)^2} = \frac{\pi}{4} \Rightarrow \sum_{n=1}^{\infty} \frac{1}{(2n-1)^2} = \frac{\pi^2}{8}$.\\ \\
\end{enumerate}
\item Let $f(x+2\pi) = f(x)$. Note $2\pi c_n = \displaystyle  \int_{-\pi}^{\pi} f(x) e^{-inx} dx = \displaystyle  \int_{0}^{2\pi} x e^{-inx} dx=-\frac{xe^{-inx}}{in}\bigg|_{0}^{2\pi} + \int_{0}^{2\pi} \frac{e^{-inx}}{in} dx= \frac{i2\pi e^{-i2n\pi}}{n}+\frac{e^{-inx}}{n^2}\bigg|_{0}^{2\pi}=\frac{i2\pi e^{-i2n\pi}}{n}+\frac{e^{-i2n\pi}}{n^2} - \frac{1}{n^2} = \frac{i2\pi}{n}$, since $e^{-i2n\pi}=1$. So $|c_n|^2 = \frac{1}{n^2}$ for $n\neq 0$. Note $c_0=\pi$, the average value of $f$, and that $|c_{n}^2|=|c_{-n}^2|$. So $\pi^2 + 2\displaystyle\sum_{n=1}^{\infty} \frac{1}{n^2} =\frac{1}{2\pi}\int_{-\pi}^{\pi} |f(x)|^2 dx = \frac{1}{2\pi}\int_{0}^{2\pi} x^2 dx = \frac{4\pi^2}{3}$ and thus $\displaystyle\sum_{n=1}^{\infty} \frac{1}{n^2} = \frac{\pi^2}{6}$. \\
\item $2 \pi c_n = \displaystyle  \int_{-\pi}^{\pi} (\pi-|x|)^2 e^{-inx} dx = \int_{0}^{\pi} (\pi-x)^2 e^{-inx} + (\pi-x)^2 e^{inx}dx = 2\int_{0}^{\pi} (\pi-x)^2 \cos nx dx= 2\int_{0}^{\pi} x^2 \cos (n\pi-nx) dx = 4x \frac{\cos(n\pi-nx)}{n^2}\bigg|_{0}^{\pi}=\frac{4\pi}{n^2}$ so $c_n=\displaystyle \frac{2}{n^2}$ for $n\neq 0$. $c_0 = \displaystyle \frac{1}{\pi}\int_{0}^{\pi} x^2 = \frac{\pi^2}{3}$. Note $c_n=c_{-n}$ so $c_n e^{-in x} + c_{-n} e^{inx} = 2 c_n \cos nx = \frac{4}{n^2} \cos nx$. So $f(x) = c_0 + \displaystyle\sum_{n=1}^{\infty} c_n e^{-in x} + c_{-n} e^{inx} = \displaystyle \frac{\pi^2}{3} + \sum_{n=1}^{\infty} \frac{4}{n^2} \cos nx$. Plugging in $x=0$ gives $\pi^2 = \displaystyle \frac{\pi^2}{3} + \sum_{n=1}^{\infty} \frac{4}{n^2}$ which rearranges into the first result. By Parseval's Theorem and similar integration tricks, we have $\displaystyle \frac{\pi^4}{9} + 2\sum_{n=1}^{\infty} \frac{4}{n^4}=\frac{1}{\pi}\int_{0}^{\pi} x^4 dx = \frac{\pi^4}{5}$ which simplifies into the second result.\\
\item Note $\sum \sin (n+\frac{1}{2}) x = \Im(\sum e^{i(n+\frac{1}{2})x})=\Im(e^{\frac{ix}{2}} \sum e^{inx})=\Im(e^{\frac{ix}{2}} \cdot \frac{1-e^{ix(N+1)}}{1-e^{ix}})=\Im(\frac{1-\cos (N+1)x - i \sin (N+1) x}{e^{-\frac{ix}{2}} - e^{\frac{ix}{2}}})=\frac{1-\cos(N+1)x}{2\sin \frac{x}{2}}$, so $\sum D_n (x) = \frac{1-\cos(N+1)x}{2\sin^2 \frac{x}{2}}=\frac{1-\cos(N+1)x}{1- \cos x}$ and $K_N$ follows. \\
\begin{enumerate}[(a)]
\item The fact that $\cos x \leq 1$ makes it clear $K_N \geq 0$. \\
\item $\displaystyle\int_{-\pi}^{\pi} D_N (x) dx = \sum_{n=-N}^{N} \int_{-\pi}^{\pi} e^{inx} dx =\sum_{n=-N}^{N} \frac{2\sin\pi n}{n} = 2\pi$ since it is $0$ everywhere except at $n=0$ which is calculated easily. So $\displaystyle \int_{-\pi}^{\pi} \sum_{n=0}^{N} D_n (x) dx =  \sum_{n=0}^{N} \int_{-\pi}^{\pi} D_n (x) dx = \sum_{n=0}^{N} 2\pi = 2\pi (N+1)$. The integral involving $K_N (x)$ follows easily.\\
\item $1-\cos(N+1)x \leq 2$ and $1-\cos x \geq 1 - \cos \delta $ so $\displaystyle \frac{1-\cos(N+1)x}{1- \cos x} \leq \frac{2}{1-\cos \delta}$ and the inequality involving $K_N (x)$ follows.\\ \\
By 8.13, $s_N(f;x) = \displaystyle\frac{1}{2\pi}\int_{-\pi}^{\pi} f(x-t) D_N(t) dt$. So $\displaystyle \sigma_N (f;x) =\frac{1}{N+1} \sum_{n=0}^{N} \frac{1}{2\pi}\int_{-\pi}^{\pi} f(x-t) D_n(t) dt=\frac{1}{2\pi}\int_{-\pi}^{\pi}f(n-t) \sum_{n=0}^{N} \frac{D_n (t)}{N+1} dx = \frac{1}{2\pi}\int_{-\pi}^{\pi}f(n-t) K_N (x) dx$.\\ \\ \\
$|\sigma_n (x) - f(x) | \leq \frac{1}{2\pi} \int_{-\pi}^{\pi} |f(x-t) -f(x)|K_n(t) dt$. $\forall \epsilon$ $\exists \delta$ s.t. $|x-y| < \delta \Rightarrow |f(x)-f(y)| < \epsilon$ since $f$ is continuous on a compact set, hence uniformly continuous. Note $f$ must also be bounded by a number $M$. So splitting our inequality gives us $|\sigma_n (x) - f(x) | \leq \displaystyle\frac{1}{2\pi} \int_{|t|<\delta} |f(x-t)-f(x)| K_n (t) dt + \frac{1}{2\pi} \int_{|t|\geq\delta} |f(x-t)-f(x)| K_n (t) dt \leq \frac{\epsilon}{2\pi}\int_{-\pi}^{\pi} K_n(t)+ \frac{1}{2\pi} \int_{|t|\geq\delta} 2M \cdot\frac{1}{N+1}\cdot \frac{2}{1-\cos \delta} dt <  \epsilon + 2M\cdot\frac{1}{n+1}\cdot\frac{2}{1-\cos \delta} < \epsilon + \epsilon = 2\epsilon$ where we can choose $n$ large enough to guarantee the second $\epsilon$, hence $\sigma_N (f; x) \rightarrow f(x)$ uniformly on $[-\pi, \pi]$.  
\end{enumerate}
\end{enumerate}
\end{document}