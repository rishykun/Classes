\documentclass{article}
\usepackage[margin=1.0in]{geometry}
\usepackage{amsmath}
\usepackage{amsfonts}
\usepackage{enumerate}
\setlength\parindent{0pt}
\begin{document}
\setlength{\abovedisplayskip}{0pt}
\setlength{\belowdisplayskip}{0pt}
\setlength{\abovedisplayshortskip}{0pt}
\setlength{\belowdisplayshortskip}{0pt}
\title{18.100B - Problem Set 6}
\author{Rishad Rahman}
\date{}
\maketitle
\begin{enumerate}
\setcounter{enumi}{5}
\item
\begin{enumerate}[(a)]
\item $s_k=\displaystyle\sum_{n=1}^{k} a_n=\displaystyle\sum_{n=1}^{k} \sqrt{n+1}-\sqrt{n}=\sqrt{k+1}-1$. Note $s_k$ is monotonically increasing and does not have an upper bound since we can always find $k$ such that $\sqrt{k+1}-1 > N$ for $N\in\mathbb{R}$. Therefore $s_n$ diverges.
\item $\displaystyle 0 < a_n=\frac{\sqrt{n+1}-\sqrt{n}}{n}=\frac{1}{n(\sqrt{n+1}+\sqrt{n})} < \frac{1}{n\cdot\sqrt{n}} = \frac{1}{n^{\frac{3}{2}}}$. $\displaystyle\sum_{n=1}^{\infty} \frac{1}{n^{\frac{3}{2}}}$ converges so $s_k$ converges as well by comparison.
\item $\displaystyle\limsup_{n\rightarrow\infty} \sqrt[n]{|a_n|} = \displaystyle\limsup_{n\rightarrow\infty} \sqrt[n]{n} - 1 = 0 < 1$ so by the root test $s_k$ converges. **Note that $\displaystyle\limsup_{n\rightarrow\infty} \sqrt[n]{n} - 1 = 0$ since $\displaystyle\lim_{n\rightarrow\infty} \sqrt[n]{n} - 1 = 0$ so the set of all subsequential limits is ${0}$. 
\item $\displaystyle\limsup_{n\rightarrow\infty} \sqrt[n]{|a_n|}=\displaystyle\limsup_{n\rightarrow\infty} \sqrt[n]{\left|\frac{1}{1+z^n}\right|}<\displaystyle\limsup_{n\rightarrow\infty} \frac{1}{|z|}=\frac{1}{|z|}$. The series diverges if $\displaystyle\frac{1}{|z|}>1\rightarrow |z|<1$. The series converges otherwise if $\displaystyle\frac{1}{|z|}<1\rightarrow |z|>1$. If $|z|=1$ then $a_n=\displaystyle\frac{1+\bar{z}^n}{|1+z^n|^2}>\frac{1+\bar{z}^n}{4}$ since $|1+z^n|^2 < (1+|z|^n)^2 = 4$ by Triangle Inequality. If $z=\cos\theta+i\sin\theta$ then $\bar{z}^n=\cos n\theta - i\sin n\theta$. So  $\displaystyle \Re(a_n)>\frac{1+\cos n\theta}{4}=\frac{\cos^2 \frac{n\theta}{2}}{2}>0$ but $\displaystyle\lim_{n\rightarrow\infty}\frac{\cos^2 \frac{n\theta}{2}}{2} \neq 0$ so the series must diverge and therefore $\Re(a_n)$ diverges by comparison so $a_n$ also diverges.\\
\end{enumerate}
\item By Schwarz Inequality $\left(\displaystyle\sum_{n=1}^{k} \frac{1}{n^2} \right)\left(\displaystyle\sum_{n=1}^{k} a_n \right)=\left(\displaystyle\sum_{n=1}^{k} \frac{1}{n^2} \right)\left(\displaystyle\sum_{n=1}^{k} \sqrt{a_n}^2 \right)\geq \displaystyle\sum_{n=1}^{k} \frac{\sqrt{a_n}}{n}$. But the left hand side is the Cauchy product of two series with one of them absolutely convergent, specifically $\displaystyle\sum_{n=1}^{k} \frac{1}{n^2}$, therefore the product is convergent. Note that the LHS monotonically increases (since the terms of the series are all positive) to its limit so $s_k=\displaystyle\sum_{n=1}^{k} \frac{\sqrt{a_n}}{n}$ is bounded. But $s_k$ is also monotonically increasing and therefore has a limit and thus the series converges.
\\
\\
\setcounter{enumi}{8}
\item
\begin{enumerate}[(a)]
\item $\displaystyle\limsup_{n\rightarrow\infty} \sqrt[n]{|n^3|}=\limsup_{n\rightarrow\infty} \sqrt[n]{n}^3 =1$ as $\displaystyle\lim_{n\rightarrow\infty} \sqrt[n]{n}^3 = 1^3=1$, so $R=1$.
\item $\displaystyle\limsup_{n\rightarrow\infty} \sqrt[n]{\left|\frac{2^n}{n!}\right|}=\limsup_{n\rightarrow\infty} \frac{2}{\sqrt[n]{n!}}=0$ since $\sqrt[n]{n!}\rightarrow \infty$. To show this: $\log \sqrt[n]{n!} = \displaystyle\frac{1}{n}\displaystyle\sum_{k=1}^{n}\log(k)$. Suppose $n=2^j-1$. Then $\displaystyle\frac{1}{n}\displaystyle\sum_{k=1}^{n}\log(k) > \displaystyle\frac{1}{2^j-1}\displaystyle\sum_{k=1}^{j-1}2^{k}\log(2^k) = \displaystyle\frac{1}{2^j-1}\displaystyle\sum_{k=1}^{j-1}k2^{k}\log(2)>\frac{(j-1)2^{j-1}}{2^j-1}\log{2} > \frac{j-1}{2}\log{2} \rightarrow \infty$ as $j\rightarrow \infty$. Note the first inequality was established by taking the highest power of $2$ lower than each term of the sum, i.e.  $\log (2^j-k) \geq \log 2^{j-1}$ for $0<k\leq2^{j-1}$, and then summing. We have shown $a_{2^k}$ goes to infinity. But $\sqrt[n+1]{(n+1)!}>\sqrt[n]{n!}\Leftrightarrow (n+1)!^n > n!^{n+1}\Leftrightarrow (n+1)^n > n! \Leftarrow (n+1)^n>n^n > n!$ so $a_n$ is monotonically increasing and therefore $a_n$ must go infinity since $a_{2^k}$ goes to infinity. Back to our series, $R=\infty$.
\item $\displaystyle\limsup_{n\rightarrow\infty} \sqrt[n]{\left|\frac{2^n}{n^2}\right|}=\displaystyle\limsup_{n\rightarrow\infty} \frac{2}{\sqrt[n]{n}^2}=2$ since $\displaystyle\lim_{n\rightarrow\infty} \frac{2}{\sqrt[n]{n}^2}=\frac{2}{1^2}=2$ so $R=\displaystyle\frac{1}{2}$.
\item $\displaystyle\limsup_{n\rightarrow\infty} \sqrt[n]{\left|\frac{n^3}{3^n}\right|}=\displaystyle\limsup_{n\rightarrow\infty} \frac{\sqrt[n]{n}^3}{3} = \frac{1}{3}$ for similar reasons as above and so $R=3$.
\\
\\
\end{enumerate}
\setcounter{enumi}{12}
\item Let the two series be $\sum{a_n}$ and $\sum{b_n}$, so $\sum{|a_n|}=A$ and $\sum{|b_n|}=B$.\\
\begin{align*}
\sum{|c_{n}|}&=|a_{0}b_{0}|+|a_{0}b_{1}+b_{1}a_{0}|+|a_{0}b_{2}+a_{1}b_{1}+a_{2}b_{0}|+...\\
&\leq |a_0b_0|+|a_0b_1|+|a_1b_0|+|a_{0}b_{2}|+|a_{1}b_{1}|+|a_{2}b_{0}|+...\\
&=\sum{|a_n|}\sum{|b_n|}=AB.
\end{align*}
Therefore $\sum{|c_{n}|}$ is bounded and obviously it is monotonically increasing so it converges.
\setcounter{enumi}{15}
\\
\\
\item 
\begin{enumerate}[(a)]
\item Note $x_n > \sqrt{\alpha}$ or $\alpha < x_n^2$ for all $n$ which follows by induction ($x_1 > \sqrt{a} > 0$, assume $x_n > \sqrt{a} > 0$, then $x_{n+1} = \frac{1}{2}\left(x_n+\frac{\alpha}{x_n}\right) \geq \sqrt{\alpha}$ by AM-GM). Now to show $x_n$ is monotonically decreasing... $x_{n+1} = \displaystyle\frac{1}{2}\left(x_n +\frac{\alpha}{x_n}
\right) < \displaystyle\frac{1}{2}\left(x_n + \frac{x_n^2}{x_n}\right) = x_n$, therefore $x_n$ is monotonically decreasing and bounded by $\sqrt{\alpha}$. To show that $\sqrt{\alpha}$ is the limit of the sequence, so we must show $\exists N \ \forall \epsilon: |x_{n}-\sqrt{\alpha}| < \epsilon$ for $n\geq N$. We can say $x_n < \sqrt{\alpha} + \epsilon$ since $x_n > \sqrt{\alpha}$. Suppose $x_1 < \sqrt{\alpha}+ \beta$ for some $\beta$. Then $x_2 = \displaystyle\frac{1}{2}\left(x_1 + \frac{\alpha}{x_1}\right) < \frac{1}{2}(\sqrt{\alpha}+\beta+\sqrt{\alpha}) < \frac{1}{2} (2\sqrt{\alpha}+\beta) = \sqrt{\alpha}+\frac{\beta}{2}$. Repeating this $n$ times leads to $x_n < \sqrt{\alpha}+\displaystyle\frac{\beta}{2^{n-1}}$. By the Archimedean property we can find $N$ such that $\displaystyle\frac{\beta}{2^{N-1}} < \epsilon $ so $x_N < \sqrt{\alpha}+\displaystyle\frac{\beta}{2^{N-1}} < \epsilon+\sqrt{\alpha}$. And since $x_n$ is monotonically decreasing, this is true for $n\geq N$ and so the limit is $\sqrt{\alpha}$.
\item $\epsilon_{n+1}=x_{n+1}-\sqrt{\alpha}=\displaystyle\frac{x_n}{2}+\frac{\alpha}{2x_n}-\sqrt{\alpha}=\frac{x_n^2+\alpha-2x_n\sqrt{\alpha}}{2x_n}=\frac{(x_n-\sqrt{a})^2}{2x_n}=\frac{\epsilon_n^2}{2x_n}<\frac{\epsilon_n^2}{2\sqrt{\alpha}}$ (since $x_n > \sqrt{\alpha}) $= $\displaystyle\frac{\epsilon_n^2}{\beta}$. By repeating the inequality $\epsilon_{n+1} < \displaystyle \frac{\epsilon_1^{2^n}}{\beta^{1+2+4+...2^{n-1}}}=\displaystyle \frac{\epsilon_1^{2^n}}{\beta^{2^n-1}}=\beta\left(\frac{\epsilon_1}{\beta}\right)^{2^n}$.
\item $\epsilon_1 = x_1 - \sqrt{\alpha} = 2 - \sqrt{3}$ and $\beta = 2\sqrt{3}$ so $\displaystyle\frac{\epsilon_1}{\beta}=\frac{2 - \sqrt{3}}{2\sqrt{3}}=\frac{2\sqrt{3}-3}{6}<\frac{2\sqrt{3.24}-3}{6}=\frac{2\cdot 1.8 - 3}{6}=\frac{1}{10}$. So $\epsilon_5 < 2\sqrt{3}\left(\frac{1}{10}\right)^{16}<2\sqrt{4}\cdot 10^{-16}=4\cdot 10^{-16}$ and $\epsilon_6 < 2\sqrt{3}\left(\frac{1}{10}\right)^{32} < 4\cdot 10^{-32}$ by similar algebra.
\\
\\
\end{enumerate}
\setcounter{enumi}{17}
\item Taking the limits gives $\lim x_{n+1} = \lim \frac{p-1}{p} x_n + \lim \frac{\alpha}{p} x_n^{-p+1} \Rightarrow L = \frac{p-1}{p} L + \frac{\alpha}{p} L^{-p+1} \Rightarrow p=p-1+\alpha L^{-p} \Rightarrow L = \sqrt[p]{\alpha}$ as long as the limit $L$ exists and is not $0$. Therefore we should verify it. Suppose $x_1 > \sqrt[p]{\alpha}$. By induction, $x_{n+1} = \frac{p-1}{p} x_n + \frac{\alpha}{p} \frac{1}{x_n^p} x_n < \frac{p-1}{p} x_n + \frac{\alpha}{p} \frac{1}{\alpha} x_n = x_n$ if $x_n > \sqrt[p]{\alpha} \leftrightarrow \alpha < x_n^p$ so $x_n$ is monotonically decreasing. Now using AM-GM on $p-1$ $x_n$'s and $\alpha x_n^{-p+1}$ gives $x_{n+1} \geq \sqrt[p]{x_n^{p-1}\alpha x_n^{-p+1}}=\sqrt[p]{\alpha}$ if $x_n > 0$ which follows by induction ($x_1 > 0$). So $|x_n - \sqrt[p]{\alpha}| < \epsilon \Rightarrow x_n < \sqrt[p]{\alpha} + \epsilon$. If $x_1 < \sqrt[p]{\alpha} + \beta$ for some $\beta \Rightarrow x_2 = \frac{p-1}{p} x_1 + \frac{\alpha}{p} \frac{1}{x_1^{p-1}} < \frac{p-1}{p}(\sqrt[p]{\alpha} + \beta) +\frac{\sqrt[p]{\alpha}}{p} = \sqrt[p]{\alpha} + \frac{p-1}{p}\beta$. Repeating this $n$ times gives $x_n < \sqrt[p]{\alpha}+(\frac{p-1}{p})^{n-1}\beta$. By the Archimedean principle we can find an $N$ such that $(\frac{p-1}{p})^{N-1}\beta < \epsilon$ so $x_N < \sqrt[p]{\alpha} + \epsilon$ and since this sequence is monotonically decreasing this is true for $n\geq N$ and so we have proved the limit is $\sqrt[p]{\alpha}$.
\end{enumerate}
\end{document}