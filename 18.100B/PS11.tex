\documentclass{article}
\usepackage[margin=1.0in]{geometry}
\usepackage{amsmath}
\usepackage{amsfonts}
\usepackage{enumerate}
\setlength\parindent{0pt}
\begin{document}
\setlength{\abovedisplayskip}{0pt}
\setlength{\belowdisplayskip}{0pt}
\setlength{\abovedisplayshortskip}{0pt}
\setlength{\belowdisplayshortskip}{0pt}
\title{18.100B - Problem Set 11}
\author{Rishad Rahman}
\date{}
\maketitle
\begin{enumerate}
\setcounter{enumi}{1}
\item $\displaystyle\lim_{n\rightarrow\infty} \sup |f-f_n| = 0$ and $\displaystyle\lim_{n\rightarrow\infty} \sup |g-g_n|=0$ where the sups are taken over $E$. $\displaystyle|f+g-f_n-g_n| \leq |f-f_n| + |g-g_n| \Rightarrow \lim_{n\rightarrow\infty}\sup |f+g-f_n-g_n| \leq \lim_{n\rightarrow\infty}\sup |f-f_n| + |g-g_n| = 0$. So ${f_n+g_n}$ uniformly converges to $f+g$.\\
$\displaystyle |fg - f_n g_n|\leq |f||g-g_n| + |g_n||f-f_n|\Rightarrow \lim_{n\rightarrow\infty}\sup |fg - f_n g_n|\leq \lim_{n\rightarrow\infty}\sup |f||g-g_n| + |g_n||f-f_n|=0$ since $g_n$ is bounded and note $f$ must be bounded as well otherwise we can find $x$ s.t. $|f-f_n| \geq |f| - |f_n| \geq |f| - M_n \geq \epsilon$. So $f_n g_n$ uniformly converges to $fg$.\\
\item Take $f_n = x$ and $g_n = \frac{1}{n}$ and $E=\mathbb{R}$. Note $f_n$ uniformly converges to $x$ and that $g_n$ to $0$. $f_n g_n = \frac{x}{n}$. Note $f_n g_n \rightarrow 0$ as $n\rightarrow 0$. But $\displaystyle\lim_{n\rightarrow\infty} \sup \left|\frac{x}{n}\right| \neq 0$. \\
\item $\frac{1}{|1+n^2x|} < \frac{1}{n^2 |x|}$ so it converges by comparison for all $x > 0$. It obviously diverges if $x=0$ since then we have an infinite summation of $1$'s or when $x=-\frac{1}{n^2}$, $n\in \mathbb{N}$, since then the series wouldn't be defined. Now consider all other values when $x<0$. Let $N$ be chosen s.t. $N^2x < -1$ so in this case we can really just test the convergence of $\displaystyle\sum_{n=N}^{\infty} \frac{1}{n^2 x -1}$ where $x>0$ (change of variables from $x$ to $-x$) and now $N^2x > 1$. But this converges by comparison to $\frac{2}{n^2 x}$. Therefore the series converges absolutely on $\mathbb{R}\backslash (\{0\}\cup\{-\frac{1}{n^2}\})$, $n\in\mathbb{N}$.
\\ Let's say the interval is $[a, b]$. Obviously $(\{0\}\cup\{-\frac{1}{n^2}\})\cap [a,b] = \emptyset$, $n\in\mathbb{N}$, otherwise $f$ wouldn't converge at all $x$. Otherwise if $b>a>0$, $|a_n|=\frac{1}{|1+n^2 x|} < \frac{1}{an^2}$ so by comparison $f$ uniformly converges. If $a<b<0$, $\frac{1}{|1+n^2 x|} \rightarrow \frac{1}{n^2x - 1} < \frac{2}{n^2 b}$ which converges, when switching to nonnegative terms i.e. $x \rightarrow-x$, $[a,b]\rightarrow [-b, -a]$ and when $n^2 x >1$. But we can guarantee this by choosing $n > \sqrt{\frac{1}{b}}$ then $n^2 x >\frac{1}{b}\cdot b = 1$. For $n\leq \sqrt{\frac{1}{b}}$ we have discrete terms which are obviously bounded so by comparison we have $f$ uniformly convergent. So any interval $[a,b]$ s.t. $(\{0\}\cup\{-\frac{1}{n^2}\})\cap [a,b] = \emptyset$, $n\in\mathbb{N}$. 
\\ Obviously if any of the aforementioned points are contained within the interval we wouldn't have a converging function but let us check intervals of the form $(a, b]$ where $a=-\frac{1}{n^2}$, $n\in\mathbb{N}$, or $a=0$. Well notice that $\sup |f-f_n| = \sup |\sum \frac{1}{1+n^2x}| = \infty$ when $x$ approaches any of those forbidden numbers so thThere is no way for $f$ to uniformly converge. 
\\ $f$ converges uniformly precisely at the same $x$ when the series converges and uniform convergence implies continuity so $f$ is continuous on the same numbers.
\\ $f$ is obviously not bounded since we can get arbitrarily get close to $\infty$ when $x\rightarrow 0$.
\setcounter{enumi}{5}
\item $\sup |f-f_n|=\sup \left|\displaystyle\sum_{k=n+1}^{\infty} (-1)^k \frac{x^2+k}{k^2} \right|$. Note that the magnitude of the terms of the series monotonically decreases, since adding $1$ to the numerator does not have as large of an effect as adding $2n+1$ to the denominator. Therefore the $\sup$ is precisely the first term or $\frac{x^2+n+1}{n^2}\rightarrow 0$ as $n\rightarrow \infty$ so $f$ uniformly converges. The series obviously does not converge absolutely since it splits into $\sum \frac{x^2}{n^2} + \frac{1}{n}$, which is a sum of a convergent and divergent series so the whole series diverges. \\ \\ \\
\item I claim $f_n$ uniformly converges to $f=0$. $\sup |f-f_n| = \sup \left| \frac{x}{1+nx^2}\right|=\sup \left | \frac{1}{\frac{1}{x} + nx}\right|=\frac{1}{2\sqrt{n}}\rightarrow 0$ by AM-GM. 
\\ $f'_n (x) =\frac{1-nx^2}{(nx^2+1)^2}\rightarrow 0$ as $n\rightarrow\infty$ when $x\neq 0$ and $f'(0)=0$ which checks. But $f'_n (0) = 1$ so the statement is true for all $x\neq 0$.\\
\setcounter{enumi}{9}
\item Claim: $f$ is discontinuous on $\mathbb{Q}$. First note $f_n$, the partial sums, is uniformly convergent since $0 \leq (nx) < 1$ so we just compare to $\frac{1}{n^2}$. Note when $x$ is irrational we can never find an $n$ such that $(nx)=0$ so $\frac{(nx)}{n^2}$ is continuous and so the partial sums are continuous and by 7.12 $f$ is therefore continuous as well. When $x\in\mathbb{Q}$ we can find $n$ s.t. $(nx) = 0$ but note it is discontinuous at that $x$ since it approaches $1$ from the left. Because $f(x) = \displaystyle\lim_{t\rightarrow x} \sum_{n=1}^{\infty} \frac{(nt)}{n^2}=\lim_{n\rightarrow \infty} \lim_{t\rightarrow x} \sum_{k=1}^{n} \frac{(kt)}{k^2}$ but the latter obviously does not exist since the right and left hand limits differ when $(nx)=0$ so $f$ is discontinuous on $\mathbb{Q}$ which we know is a countable dense set.
\\ Note $f_n$ is Riemann integrable since it is a finite sum of Riemann integrable functions (since they have finitely many jump discontinuities on any bounded interval) and so $f$ must be Riemann integrable by Theorem 7.16.\\
\setcounter{enumi}{11}
\item $\displaystyle\lim_{n\rightarrow\infty} \int_{0}^{\infty} f_n(x) dx = \lim_{n\rightarrow\infty} \lim_{c\rightarrow\infty} \int_{0}^{c} f_n(x) dx$. We just need to show that $\int f_n (x) dx$ converges uniformly to $\int f(x) dx$ on $(0, \infty)$ then we can exchange the order of the limits by 7.11 and we are done, note the process is similar when taking $0$ as the limit point instead of $\infty$. Let us say we have a bounded interval $[a, b]$. Since $f_n$ uniformly converges to $f$, $\forall\epsilon>0$ $\exists N$ s.t. for $n>N$ $\displaystyle\left| \int_{a}^{b} f-f_n dx \right| \leq  \int_{a}^{b} |f-f_n| dx < (b-a) \epsilon$ and so we have shown the integral uniformly converges as well so we can obtain the result.\\
\setcounter{enumi}{13}
\item $\Phi$ is continuous since $x(t)$ and $y(t)$ are continuous by $7.10$. 
\\ $3^k t_0 = \displaystyle\sum_{i=1}^{\infty} 3^{k-i-1} (2a_i)\equiv \sum_{i=1}^{\infty} 3^{-i}(2a_{k+i-1}) \mod 2$. If $a_k = 0$ then the sum is in the interval $[0, \frac{1}{3}]$ so $f(3^k t_0) =0=a_k$. Similarly if $a_k = 1$ then the sum is in the interval $[\frac{2}{3}, 1]$ and $f(3^k t_0) = 1 = a_k$. Substituting $2n-1$ and $2n$ for $k$ and then substituting into the definitions for $x(t)$ and $y(t)$ gives the result.
\end{enumerate} 
\end{document}