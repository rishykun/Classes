\documentclass{article}
\usepackage[margin=1.0in]{geometry}
\usepackage{amsmath}
\usepackage{amsfonts}
\usepackage{enumerate}
\setlength\parindent{0pt}
\begin{document}
\setlength{\abovedisplayskip}{0pt}
\setlength{\belowdisplayskip}{0pt}
\setlength{\abovedisplayshortskip}{0pt}
\setlength{\belowdisplayshortskip}{0pt}
\title{18.100B - Problem Set 7}
\author{Rishad Rahman}
\date{}
\maketitle
\begin{enumerate}
\item No $f$ need not be continuous. Take $f(x) = x$ except when $x=0$, otherwise $f(0)=1$. It is not hard to see $\forall x$ $\displaystyle\lim_{h\rightarrow 0} f(x+h) - f(x-h) = \displaystyle\lim_{h\rightarrow 0} 2h = 0$ since we can always find a ball around $x\neq0$ with $r< d(x, 0)$ and in that ball $f(x)=x$ and if $x=0$ then $x+h\neq 0$ and $x-h\neq 0$. So this function satisfies the condition but is obviously not continuous at $x=0$ since $\displaystyle\lim_{x\rightarrow 0} f(x) = 0$.
\item If $x\in E$ then $f(x)\in f(E)\subset \overline{f(E)}$. Suppose $x\in E'$, then $\forall \epsilon_1 > 0$ $\exists y \in E$ s.t $d(x,y) < \epsilon_1$. Note $f(y)\in f(E)$. Since $f$ is continuous, $\forall\epsilon_2 > 0$ $\exists\delta$ s.t $d(f(x), f(x^*))<\epsilon_2$ $\forall x^*$ s.t. $d(x, x^*) < \delta$. Take $\epsilon_1 = \delta$, then $\exists y\in f(E)$ and $\delta$ s.t $d(x,y)<\delta\Rightarrow d(f(x),f(y)) <\epsilon_2$ $\forall \epsilon_2 > 0$ $\Rightarrow f(x)$ is a limit point of $f(E)$ so $f(x) \in f(E)' \subset \overline{f(E)}$. Therefore $f(\overline{E})\subset\overline{f(E)}$. Let $f(x) = e^{-x}$ where $x\in X = (0, +\infty)$ and $E=\mathbb{N}$. Then $\overline{E} = \mathbb{N}$ so $f(E) = f(\overline{E})= e^{-n}$, $n\in \mathbb{N}$, but $\overline{f(E)}=e^{-n} \cup \{0\}$for $n\in \mathbb{N}$. Therefore $f(\overline{E})$ is a proper subset of $\overline{f(E)}$.
\setcounter{enumi}{3}
\item If $x\in E$ then $f(x) \in f(E)$. Otherwise suppose $x\in X$ is a limit point of $E$. Note that $f(x) \in f(X)$ and we showed in problem 2 that $f(x)$ is a limit point of $f(E)$. So now we have established $f(E)$ is dense in $f(X)$. Note everything for $f$ goes as well for $g$. Suppose $p\in E^c$. Since $f(E)$ is dense in $f(X)$, $f(p)$ is a limit point of $f(E)$ and so $\forall\epsilon > 0 \ \exists f(x)\in f(E)$ s.t. $d(f(p), f(x)) < \epsilon$. So $d(g(x), g(p)) <\epsilon$ as well. Note $x\in E$ so $f(x)=g(x)$. Therefore $d(f(x), g(p)) < \epsilon$, adding the two triangles together and then using the Triangle Inequality gives $d(f(p), g(p)) < 2\epsilon$ $\forall \epsilon > 0$ which forces $d(f(p), g(p)) = 0$ so $f(p) = g(p)$. 
\item Note $E^c$ is open so it can be formed by the union of a most countable number of disjoint segments, $\cup (a_k, b_k)$. Let $g(x)=f(x)$ if $x\in E$, otherwise $g(x) = f(a_i) + \frac{f(b_i) - f(a_i)}{b_i-a_i} \cdot (x-a_i)$ where $a_i < x < b_i$. $g$ is obviously continuous on the line segments or on an interior point of $E$. Otherwise $\displaystyle \lim_{x\rightarrow b_i^{-}} g(x) = f(a_i) + f(b_i) - f(a_i) = f(b_i)=\displaystyle \lim_{x\rightarrow b_i^{+}} g(x)$ so $g(x)$ is continuous on $R^1$. Take $f(x)=\frac{1}{x}$ on $(0,+\infty)$. That set is open but there is no way to assign a value to $g(0)$ to create a continuous extension in $R^1$. If $f(x)=<f_1 (x), f_2 (x), ... , f_n (x)>$ is continuous, then each component, $f_k (x)$, must be continuous as well. Let a continuous extension of $f_k (x)$ be $g_k (x)$, then the continuous extension of $f(x)$ is $<g_1 (x), g_2 (x), ... , g_n (x)>$. 
\setcounter{enumi}{7}
\item Lemma: If $p,q\in [c, c+\delta]$ where $p,q\in E$ and $|f(p)-f(q)| < \epsilon \ \forall p,q$ then $f$ is bounded on $[c, c+\delta]$. \
Proof: Let $E\cap [c, c+\delta] = F$. Assume $f$ was unbounded on $F$. This means $\forall p\in F$ and $\forall \epsilon$ we can find $q\in F$ s.t. $|f(p)| > \epsilon + |f(q)|$. But $|f(p)-f(q)| \geq |f(p)| - |f(q)| > \epsilon$, a contradiction so $f$ is bounded on $F$. \\
Suppose $\inf E = M$ and $\sup E = N$ where $M, N\in\mathbb{R}$ which exist since $E$ is bounded in $\mathbb{R}$. Since $f$ is uniformly continuous,  $\exists$ a $\delta$ neighborhood such that for all $p,q$ in $E$ if $|p-q| <\delta$ then $|f(p)-f(q)| < \epsilon$. Note that this satisfies the conditions of the lemma. Divide $[M, N]$ into the sub-intervals $[M, M+\delta], [M+\delta, M+2\delta], ..., [M+(n-1)\delta, M+n\delta]$ where $N\in [M+(n-1)\delta, M+n\delta]$. By the lemma, $f$ is bounded on each of those sub-intervals and therefore $f$ must be bounded on the union of those intervals, $E$, as well by taking the maximum upper bound. If $E$ was not bounded, we could just take $f(x)=x$ on $E=\mathbb{R}$ which is uniformly continuous but not bounded.
\setcounter{enumi}{13}
\item Either $f(x) > x$ or $f(x) <x$ for all $x\in [0,1]$, otherwise we could guarantee $f(x)=x$ by the Intermediate Value Theorem (Let $g(x) = f(x) - x$, if $g(a) > 0$ and $g(b) <0$ there must be a $c$ in between such that $g(c)=0\Rightarrow f(x)=x$). If $f(x) > x > 0$ for $x>0$ and $f(0) \neq 0 \Rightarrow 0\notin I$. If $f(x) < x < 1$ for $x<1$ and $f(1)\neq 1 \Rightarrow 1\notin I$. This is a contradiction so we must have $f(x)=x$ for some $x\in[0,1]$. \\
\item Suppose $f$ was not monotonic. Then there exists an interval $(a,b)$ with $c\in (a, b)$ s.t. $f$ is monotonically increasing on $(a,c]$ and decreasing on $[c, b)$ or vice versa but the proof for the other case would be analogous. So $f(c)$, which exists since $f$ is continuous, would be a maximum on $(a, b)$ so the range of $f$ on $(a, b)$ would be of the form $(k, f(c)]$. But this set is not open even though $(a, b)$ is open so this contradicts the fact $f$ is a continuous open mapping, therefore $f$ must be monotonic. \
\item $f(x)=[x]$ has discontinuities of the first kind on $\mathbb{Z}$. Let's first show it is continuous on $\mathbb{R}\backslash \mathbb{Z}$. Suppose $n < p < n+1$ where $n\in\mathbb{Z}$ so $f(p)=n$. Pick $\delta = \min(d(p, n), d(p, n+1))$. Then if $d(x,p) < \delta \Rightarrow x\in (n, n+1) \Rightarrow f(x)=n \Rightarrow d(f(p), f(x)) = 0 < \epsilon \ \forall \epsilon >0$ so $f$ is continuous in $\mathbb{R}\backslash\mathbb{Z}$. However if $p=n$ then $f(p)=n$ but $\displaystyle\lim_{x\rightarrow p^{-}} f(x) = n-1$ so $f$ is discontinuous there and since it is a jump, it is of the first kind.\
$f(x)=(x)$ also has discontinuities of the first kind on $\mathbb{Z}$, and is continuous everywhere else. However this is easy to show since $f(x)=(x)=x-[x]$ is a sum of two continuous functions if $x\notin\mathbb{Z}$ and is a sum of a continuous and discontinuous function if $x\in\mathbb{Z}$. The properties of limits will guarantee continuity in the former but the left and right hand limits will still disagree in the latter. \
\setcounter{enumi}{21}
\item We will need the results of 20. \\
20a): $\rho_{E} (x) = \displaystyle\inf_{z\in E} d(x, z)\geq 0$. If $x\in E$ then $d(x, x)=0$ so $\rho_{E} (x) = 0$. If $x\in E'$, then $\forall\epsilon$ $\exists z\in E$ s.t. $d(x, z) < \epsilon$. So $0 \leq \displaystyle\inf d(x, z)\leq\displaystyle\inf \epsilon = 0$ so $\rho_{E} (x) = 0$. Now suppose $\displaystyle\inf_{z\in E} d(x, z) = 0$ so $\forall \epsilon$ $\exists z\in E$ s.t $0\leq d(x,z) < \epsilon \Rightarrow x\in E'\subset\overline{E}$. \\
20b): From the hint, which follows from taking the infimum of both sides of the Triangle Inequality, $\rho_{E} (x) \leq d(x,y) +\rho_{E} (y)$ and by a symmetric argument $\rho_{E} (y) \leq d(x,y) +\rho_{E} (x)$ so $|\rho_{E} (x) - \rho_{E} (y)| \leq d(x,y)$. If $d(x,y)<\delta = \epsilon$ then $d(\rho_{E} (x), \rho_{E} (y)) < \epsilon$ which implies $\rho_{E} (x)$ is uniformly continuous. \\
Back to 22, if $f(p) = 0$ then $\rho_A (p) = 0 \implies p\in \overline{A} = A$ by 20a. Note if $\rho_A (p) + \rho_B (p) =  = 0$, then $\rho_A (p) = \rho_B (p) = 0 \implies p\in A \cap B = \phi$ so this is impossible. If $f(p) = 1$ then $\rho_B (p) = 0$ so $p\in B$ for precisely the same reason. $f$ is continuous since it is a quotient of two continuous functions, implied by 20b, where the denominator is never $0$. $0\leq \rho_A (p), \rho_B (p)$ so $0\leq f(p) \leq 1 \Rightarrow$ range of $f$ is in $[0, 1]$. Since $f$ is continuous and $[0, \frac{1}{2})$, $(\frac{1}{2}, 1]$ are open in $[0, 1]$, their preimages are open as well so $V$ and $W$ are open. They are also disjoint since $f(p)$ cannot be in both intervals at the same time by the definition of a function. $p\in A \Rightarrow f(p) = 0 \Rightarrow p\in f^{-1} (0) \subset V$ and similarly $p\in B \Rightarrow f(p) = 1 \Rightarrow p \in f^{-1} (1) \subset W$. \\
\item As $x\rightarrow p^{-}$, $f(\lambda p + (1-\lambda)x)\rightarrow f^{-}(p)$, and $\lambda f(p) + (1-\lambda)f(x) \rightarrow \lambda (f(p)-f^{-} (p))+f^{-} (p)$ where $f^{-}$ represents the left handed limit. So we have $f^{-}(p)\leq \lambda (f(p)-f^{-} (p))+f^{-} (p)\Rightarrow 0 \leq \lambda ((f(p)-f^{-} (p)))$ and $\lambda > 1$ so $f(p) \geq f^{-} (p)$. Similarly $f(p) \geq f^{+} (p)$. However if we let $\lambda x + (1-\lambda)y = p$ so $f(p)\leq \lambda f(x) +(1-\lambda)f(y)$. Note as $x\rightarrow p^{-}$, $y\rightarrow p^{+}$. So taking the left handed limit with respect to $x$, $f(p) \leq \lambda f^{-} (x) +(1-\lambda) f^{+} (x)$. So similarly $f(p) \leq \lambda f^{+} (x) +(1-\lambda) f^{-} (x)$. Adding the two inequalities gives $2f(p) \leq f^{-} (x) + f^{+} (x)$ but $f^{-}(p), f^{+} \leq f(p)$ so this forces $f^{-}(p)= f^{+} (p)=f(p)$ and therefore the function is continuous.\\
Let this function be of the form $h(x)=g(f(x))$ where $f$ and $g$ are convex and $g$ is increasing.
\begin{flalign*}
h(\lambda x + (1-\lambda)y)=g(f(\lambda x + (1-\lambda)y)) &\leq g(\lambda f(x) + (1-\lambda) f(y)) \ \text{since $f$ is convex and $g$ is increasing} \\
&\leq \lambda g(f(x))+(1-\lambda)g(f(y)) \ \text{since $g$ is convex} \\ 
&=\lambda h(x) + (1-\lambda)h(y) \ \text{therefore $h$ is convex}
\end{flalign*}
Let $\lambda = \displaystyle\frac{t-s}{u-s}$, $x=u$, $y=s$, then\\
\begin{flalign*}
f\left(\displaystyle\frac{t-s}{u-s} u +\displaystyle\frac{u-t}{u-s} s\right)=f(t) &\leq\displaystyle \frac{t-s}{u-s}f(u) + \frac{u-t}{u-s}f(s)=\displaystyle \frac{t-s}{u-s}f(u) + \frac{s-t}{u-s}f(s) + f(s)\\
&=\displaystyle\frac{t-s}{u-s} (f(u)-f(s)) + f(s) \ \text{so} \ \displaystyle\frac{f(t)-f(s)}{t-s} \leq \frac{f(u)-f(s)}{u-s}.
\end{flalign*}
Let $\lambda = \displaystyle\frac{u-t}{u-s}$, $x=s$, $y=u$, then \\
\begin{flalign*}
f\left(\displaystyle\frac{u-t}{u-s} s +\displaystyle\frac{t-s}{u-s} u\right)=f(t) &\leq\displaystyle \frac{u-t}{u-s}f(s) + \frac{t-s}{u-s}f(u)=\displaystyle \frac{u-t}{u-s}f(s) + \frac{t-u}{u-s}f(u) + f(u)\\
&=\displaystyle\frac{u-t}{u-s} (f(s)-f(u)) + f(u) \ \text{so} \ \displaystyle\frac{f(u)-f(t)}{u-t} \geq \frac{f(u)-f(s)}{u-s}.
\end{flalign*}
\item First of all note that the conditions for a convex function are satisfied for $\lambda = \frac{m}{2^k}$ which follows by induction. If $k=1$, $\lambda = 0, 1, \frac{1}{2}$ which hold by the given conditions. Now suppose $f(\frac{m}{2^n} x+\frac{2^n-m}{2^n} y)\leq \frac{m}{2^n} f(x) + \frac{2^n-m}{2^n} f(y)$. Then $f(\frac{m}{2^{n+1}} x+\frac{2^{n+1}-m}{2^{n+1}} y)\leq \frac{1}{2}(f(\frac{m}{2^n} x+\frac{2^n-m}{2^n} y) +f(y))\leq \frac{m}{2^{n+1}} f(x) + \frac{2^{n+1}-m}{2^{n+1}} f(y)$ which completes the induction. Now $f$ is continuous and all real numbers have a binary expansion. This means there is a sequence of $\lambda_i$s which converge to any $\lambda\in(0, 1)$ so by taking limits we can say $f(\lambda x + (1-\lambda) y) = \displaystyle\lim_{i\rightarrow\infty} f(\lambda_i x + (1-\lambda_i) y) \leq \displaystyle\lim_{i\rightarrow\infty} \lambda_i f(x) + (1-\lambda_i) f(y) = \lambda f(x) + (1-\lambda) f(y)$.
\end{enumerate}
\end{document}