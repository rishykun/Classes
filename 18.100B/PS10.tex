\documentclass{article}
\usepackage[margin=1.0in]{geometry}
\usepackage{amsmath}
\usepackage{amsfonts}
\usepackage{enumerate}
\setlength\parindent{0pt}
\begin{document}
\setlength{\abovedisplayskip}{0pt}
\setlength{\belowdisplayskip}{0pt}
\setlength{\abovedisplayshortskip}{0pt}
\setlength{\belowdisplayshortskip}{0pt}
\title{18.100B - Problem Set 10}
\author{Rishad Rahman}
\date{}
\maketitle
\begin{enumerate}
\setcounter{enumi}{6}
\item
\begin{enumerate}[(a)]
\item $\displaystyle cm_c\leq \int_{0}^{c} f(x) dx \leq cM_c$ where $m_c$ and $M_c$ is the inf and sup on $[0, c]$. This means $\displaystyle cm_c\leq \int_{0}^{1} f(x) dx - \int_{c}^{1} f(x) dx\leq cM_c$ $\forall c$. Taking $c\rightarrow 0$ gives $\displaystyle\int_{0}^{1} f(x) dx - \lim_{c\rightarrow 0} \int_{c}^{1} f(x) dx=0$\\
\item $f(x) = (-1)^{\lfloor \frac{1}{x} \rfloor + 1} \lfloor \frac{1}{x}\rfloor$ for $x\in (0, 1]$. It is not difficult to see that if $\displaystyle \frac{1}{n+1} \leq c < \frac{1}{n}$ then $\displaystyle \int_{c}^{1} f(x) dx = \frac{1}{n}- c + \sum_{k=1}^{n} (-1)^{k+1} k\left(\frac{1}{k}-\frac{1}{k+1}\right)$, by partitioning $[c, 1]$ with the integer reciprocals in the interval. Note as $c\rightarrow 0$, $n\rightarrow\infty$ so $\displaystyle \lim_{c\rightarrow 0} \int_{c}^{1} f(x) dx = \sum_{k=1}^{\infty} (-1)^{k+1} k\left(\frac{1}{k}-\frac{1}{k+1}\right) = \sum_{k=1}^{\infty} \frac{(-1)^{k+1}}{k+1}$ which converges by the alternating series test but does not converge absolutely, which is what happens when we integrate $|f|$, since then it would be a p-series with $p=1$. 
\end{enumerate}
\item Suppose $\displaystyle\sum_{n=1}^{\infty} f(n)$ converges. Partition $[1, b]$ with $x_n= n$ and so that $x_k =\lfloor b\rfloor$. Note $M_n = f(n)$ since $f$ is monotonically decreasing. Then $\displaystyle\int_{1}^{b} f(x) dx \leq M_n \Delta x_n = (b-k)f(k) + \displaystyle\sum_{n=1}^{k-1} f(n)\leq \displaystyle\sum_{n=1}^{k} f(n)$ and since $f\geq 0$, the integral increases as $b$ increases. Since this is bounded and increasing, the limit must exist so $\displaystyle\int_{1}^{\infty} f(x) dx$ converges. The other direction is analogous, we jut have to use $m_n = f(n+1)$ instead and reverse the inequality signs to get $\displaystyle\int_{1}^{\infty} f(x) dx \geq \displaystyle\sum_{n=2}^{\infty} f(n)$ and by the same argument we can conclude that the series converges. \\
\item Suppose $f, g$ are differentiable and $f',g'\in \mathcal{R}$, then $\displaystyle\int_{0}^\infty f(x) g'(x) dx = \lim_{b\rightarrow\infty} f(b)g(b) - f(0)g(0) - \int_{0}^{b} f'(x) g(x) dx$. Since the function from integration by parts is continuous, we are allowed to take limits provided that they converge using the defintion from the previous problem. Letting $f(x) =\frac{1}{1+x}$ and $g'(x) = \cos x$ we get $\displaystyle\int_{0}^\infty \frac{\cos x}{1+x} dx = \lim_{b\rightarrow\infty} \frac{\sin b}{1+b} - 0 - \displaystyle\int_{0}^\infty \frac{-\sin x}{(1+x)^2} dx=\displaystyle\int_{0}^\infty \frac{\sin x}{(1+x)^2} dx$. The second series absolutely converges by the integral test after using the fact $|\sin x| \leq 1$. The first one does not converge absolutely since it it will be greater than a multiple of the harmonic series, using $x=2\pi N$ as partitions.\\
\\
\\
\\
\\
\item 
\begin{enumerate}[(a)]
\item By weighted AM-GM $\frac{qu^p+pv^q}{p+q}\geq \sqrt[p+q]{(uv)^{pq}}$ but $p+q=pq$ so this leads to $uv\leq \frac{u^p}{p}+\frac{v^q}{q}$.\\
\item $\displaystyle\int_{a}^{b} fg d\alpha \leq \int_{a}^{b} \frac{f^p}{p}+\frac{g^q}{q} d\alpha= \frac{1}{p}+\frac{1}{q}=1$. \\
\item Note $\left|\displaystyle\int_{a}^{b} fg d\alpha\right|\leq \displaystyle\int_{a}^{b} |f||g| d\alpha$. Suppose $c=\displaystyle\int_{a}^{b} |f|^p d\alpha$ and $d=\displaystyle\int_{a}^{b} |g|^q d\alpha$, then if $c,d\neq 0$ $1=\displaystyle\int_{a}^{b} \frac{|f|^p}{c} d\alpha$ and  $1=\displaystyle\int_{a}^{b} \frac{|g|^q}{d} d\alpha$. So by $(b)$, $\displaystyle\int_{a}^{b} \frac{|f||g|}{c^{\frac{1}{p}} d^{\frac{1}{p}}} d\alpha\leq 1$ and the result follows. If either $c$ or $d$ is $0$ this forces either $|f|$ or $|g|$ to be $0$ so the result is trivial.
\item Assume it doesn't hold for improper integrals. Then we would have LHS $>$ RHS as we approach $0$ or $\infty$. But this would mean there would have to be a neighborhood of $0$ or $\infty$ which this inequality holds and in that case we would have proper integrals and it would would break Holder's Inequality so it must be held for improper integrals as well.
\end{enumerate}
\setcounter{enumi}{12}
\item 
\begin{enumerate}[(a)]
\item Setting $t=\sqrt{u}\rightarrow dt = \frac{1}{2\sqrt{u}} du$ gives $\displaystyle f(x) = \int_{x^2}^{(x+1)^2} \frac{\sin u}{2\sqrt{u}} du$. Then setting $g'(u)=\sin u$ and $f(u)=\frac{1}{2\sqrt{u}}$ and using integration by parts, we get $f(x) = \displaystyle -\frac{\cos [(x+1)^2]}{2(x+1)} + \frac{\cos(x^2)}{2x} - \int_{x^2}^{(x+1)^2} \frac{\cos u}{4 u^{\frac{3}{2}}} du$. Since $\cos u \geq -1$, $f(x) < \displaystyle -\frac{\cos [(x+1)^2]}{2(x+1)} + \frac{\cos(x^2)}{2x} + \int_{x^2}^{(x+1)^2} \frac{1}{4 u^{\frac{3}{2}}} du=\displaystyle -\frac{\cos [(x+1)^2]}{2(x+1)} + \frac{\cos(x^2)}{2x} - \frac{1}{2(x+1)}+\frac{1}{2x}$. So $f(x) < \displaystyle\frac{1+\cos(x^2)}{2x} - \frac{1+\cos((x+1)^2)}{2(x+1)}< \frac{1}{x}$ using $-1 \leq \cos t \leq 1$. If we used the fact that $\cos u \leq 1$ we would have gotten $f(x) > \displaystyle\frac{-1+\cos(x^2)}{2x} - \frac{-1+\cos((x+1)^2)}{2(x+1)}> -\frac{1}{x}$ so $|f(x)| < \displaystyle\frac{1}{x}$. Everything is strict since $\cos t$ is never $1$ or $-1$ all the time.
\item $\displaystyle r(x) = 2x f(x) - \cos(x^2) + \cos[(x+1)^2] = -\frac{x \cos[(x+1)^2]}{x+1}+\cos(x^2) - 2x\int_{x^2}^{(x+1)^2} \frac{\cos u}{4 u^{\frac{3}{2}}} du - \cos (x^2) + \cos[(x+1)^2]=\frac{\cos[(x+1)^2]}{x+1} - 2x\int_{x^2}^{(x+1)^2} \frac{\cos u}{4 u^{\frac{3}{2}}}$. So $\displaystyle|r(x)| < \left|\frac{\cos[(x+1)^2]}{x+1}\right|+\left|-\frac{x}{(x+1)}+1\right|<\frac{1}{x+1}+\frac{1}{x+1} = \frac{2}{x+1} < \frac{2}{x}$.\\
\item Note $r(x)\rightarrow 0$ so we really just need to look at $\frac{\cos (x^2) - \cos [(x+1)^2]}{2}=\sin \left(x^2+x+\frac{1}{2}\right)\sin\left(x+\frac{1}{2}\right)$. Suppose $x=\sqrt{n\pi}$, $n\in\mathbb{N}$ then the expression turns into $\sin(n\pi+x+\frac{1}{2})\sin(x+\frac{1}{2})=(-1)^n \sin^2(x+\frac{1}{2})$. We want to show $\sqrt{n\pi}+\frac{1}{2}$ gets arbitrarily close to $\frac{\pi}{2}+2\pi m$. Note $n$ can be chosen such that $\sqrt{n\pi}+\frac{1}{2} \leq \frac{\pi}{2}+2\pi m < \sqrt{(n+1)\pi}+\frac{1}{2}$, since after doing operations on the inequality, we will get $n\leq f(m) < n+1$. So $\left|\frac{\pi}{2}+2\pi m-\sqrt{n\pi}-\frac{1}{2}\right|<\sqrt{(n+1)\pi}-\sqrt{n\pi}$ which gets arbitrarily small when $n\rightarrow\infty$, which we could see after rationalizing. Therefore $\sin \left(x^2+x+\frac{1}{2}\right)\sin\left(x+\frac{1}{2}\right)$ gets arbitrarily close to $\pm 1$ and it is obvious it can't go beyond those bounds so the upper limit is $1$ and lower limit is $-1$.\\
\item $\displaystyle\int_{0}^{N} \sin(t^2) dt =\int_{0}^{1} \sin(t^2) dt+\int_{1}^{N} \sin(t^2) dt=\int_{0}^{1} \sin(t^2) dt+\sum_{k=1}^{N-1} \int_{k}^{k+1} dt \sin (t^2) dt=\int_{0}^{1} \sin(t^2) dt+\sum_{k=1}^{N-1} f(k-1) < \int_{0}^{1} \sin(t^2) dt+\sum_{k=1}^{N-1} \frac{1}{2}\left(\frac{\cos(k^2)-\cos[(k+1)^2]}{k}+\frac{2}{k^2}\right)< \int_{0}^{1} \sin(t^2) dt+\frac{1}{2}\sum_{k=1}^{N-1} \frac{\cos(k^2)}k - \frac{\cos[(k+1)^2]}{k+1} + \frac{2}{k^2}$ Now taking $N\rightarrow\infty$, we see that the integral is bounded above since the first two parts of the series telescope to $\frac{\cos (N^2)}{N} \rightarrow 0$ and the third part is a p-series with $p=2$. We can show it is bounded below using $r(x) > \frac{-2}{x}$, hence the integral converges.
\end{enumerate}
\end{enumerate}
\end{document}